%%%%%%%%%%%%%%%%%%%%%%%%%%%%%%%%%%%%%%%%%%%%%%%%%%%%%%%%%%%%%%%%%%%%%%%%
%                                                                      %
%     File: Thesis_Resumo.tex                                          %
%     Tex Master: Thesis.tex                                           %
%                                                                      %
%     Author: Andre C. Marta                                           %
%     Last modified :  2 Jul 2015                                      %
%                                                                      %
%%%%%%%%%%%%%%%%%%%%%%%%%%%%%%%%%%%%%%%%%%%%%%%%%%%%%%%%%%%%%%%%%%%%%%%%

\section*{Resumo}

% Add entry in the table of contents as section
\addcontentsline{toc}{section}{Resumo}

Nesta tese, são calculadas as constantes NP para um campo de spin-0 propagando no espaço-tempo de Minkowski, com ênfase no comportamento próximo ao infinito espacial e ao infinito nulo. Para alcançar isso, é utilizada a estrutura do $i^0$ cilindro de Friedrich. Sob a suposição de que os dados iniciais atendem a determinadas condições de regularidade, permitindo a extensão analítica do campo para conjuntos críticos, o estudo revela que as constantes NP no infinito nulo futuro $\mathscr{I}^{+}$ e no infinito nulo passado $\mathscr{I}^{-}$ são independentes uma da outra. Noutras palavras, as constantes NP clássicas em $\mathscr{I}^{\pm}$ são determinadas por partes diferentes dos dados iniciais, que são definidos em uma hipersuperfície de Cauchy.
Por outro lado, ao introduzir uma pequena generalização conhecida como constantes NP do $i^0$ cilindro, a necessidade da condição de regularidade é eliminada. Estas constantes NP modificadas fornecem quantidades conservadas em $\mathscr{I}^{\pm}$ que são exclusivamente determinadas por uma parte específica dos dados iniciais, que, por sua vez, correspondem aos termos que governam a regularidade do campo. Esta característica mostra-se fascinante no estudo de equações de evolução usando a estrutura do $i^0$ cilindro.

\vfill

\textbf{\Large Palavras-chave:} Constantes de Newman-Penrose, infinidade nula, Cilindro de Friedrich, campo de spin-0.

